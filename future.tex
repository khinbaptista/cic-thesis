\chapter{Future Work}
We have some ideas as to how this project could be improved. For example, the definition of a project description file format would be of great benefit for students, allowing them to share projects between one another and the teacher. It may also be desirable to support more types of shaders, like geometry, tesselation control, and tesselation evaluation shaders. Although the graphics hardware supports cube map samplers, Shader Tutor does not parse uniform variables defined as sampler cubes, but doing so would be useful to create certain effects, such as environment mapping. Adding lights as entities of the application, broadcasting their attributes to all shaders that required them, instead of having each shader add their own light parameters. Finally, removing and renaming resources (materials and objects) could improve the experience of students using the application.

The overall implementation could also benefit from more rigid software engineering guidelines in order to separate the core functionality from the interface used to communicate with the control window (using the "Model-View-Controller" model, for example).

This application could also be expanded to help students understand general computer graphics topics instead of just shader programming. This could be achieved if the Vulkan API configuration values in its various structures were exposed to the user in an organized manner, so that the student could tweak parameters of the entire graphics pipeline and see the changes in real time. Going even further, this could evolve into a tool to learn the Vulkan API specifically, if the user were given the chance to, for example, see the available physical devices, their features and extensions, being able to enable and disable each of them at will. Such tool could be really enlightening to anyone using the API for the first time.
