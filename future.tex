\section{Future Work}
There are some ideas as to how this project could be improved. For example, the creation os a project file format would be of great benefice for students, allowing them to share projects between one another and the teacher; It may also be desirable to support more types of shaders, like geometry, tesselation control and tesselation evaluation shaders; Cubemap sampler support would also be useful to create certain effects; Finally, allowing the user to remove materials and objects and renaming resources (materials and objects) could improve the experience of students using the application.

The overall implementation could also benefit from more rigid software engineering guidelines in order to separete the core functionality to the interface used to communicate with the control window (using the model-view-controller model, for example).

This application could also be expanded to help students understand general computer graphics topics instead of just shader programming. This could be achieved if the Vulkan API configuration values in its various structures were exposed to the user in an organized manner, so that the student could tweak parameters of the entire graphics pipeline and see the changes in real time. Going even further, this could evolve into a tool to learn the Vulkan API specifically, if the user were given the chance to, for example, see the available physical devices, their features and extensions, being able to enable and disable each of them at will. Such tool could be really enlightening to anyone using the API for the first time.
