\section{Architecture}
In this work, the most important decision of the architecture was the choice of using Vulkan for the graphics API. Vulkan is a low-overhead, cross-platform 3D graphics and compute API created by the Khronos Group. Using Vulkan has been a learning process, and the architecture chosen for this work is a reflection of that learning process.

For the purposes of this work, a Material is defined as a combination of a vertex shader and a fragment shader, along with their uniforms values. A Shader is the created from the source GLSL code provided by the user, which is then compiled into SPIR-V binary code and stored. In order to be able to preview the material, a Geometry object can be created: a geometri is a combination of vertex attributes (vertex position, normal, tangent and texture coordinates) which will be passed to the graphics pipeline.

\subsection{Shader Module}
A Shader object represents one stage of the programmable pipeline stages. In this work, it has a type, which indicate whether the shader is a vertex or a fragment shader. The type of the shader is inferred when a GLSL shader file is loaded based on the extension of the shader file, '.vert' for a vertex shader and '.frag' for a fragment shader.

The module is also capable of compiling the GLSL source code into SPIR-V, an intermediate binary file format. The compilation is done using a third-party library, libshaderc, developed by Google. The SPIR-V code is then used to create the Vulkan shader module, an object used in the pipeline creation later.

Shaders can define uniform objects containing various uniforms. The uniforms names are checked for specific names and, if their types match, their values are loaded automatically. This applies, for example, to a uniform of type "mat4" named "modelViewProjection". A similar process is adopted for vertex attributes, where their names and types are checked in order to load the correct values in the vertex buffer later on.

Possible uniform names updated automatically are: "model", "view", "projection", "modelView", "modelViewProjection", their inverse counterparts with "Inverse" as suffix, "cameraMatrix" (which is the view and projection transformations combined) and "normalMatrix" (the inverse transposed 3x3 modelView submatrix). All the matrices must be of type "mat4", except the normal matrix, which must be a "mat3" matrix.

Possible vertex attributes are: "position", "normal", "tangent" and "uv" (texture coordinates). The first three attributes may be either of type "vec3" or "vec4". Position vectors with four components will have W component equal to one; Normal vectors will have a W component equal to zero; Tangent vectors will have either 1 or -1 for its fourth component, indicating which side the binormal vector points to. Texture coordinates must be of type "vec2".

Shaders can also define image samplers of type "sampler2D" and "samplerCube" in GLSL. These will be initialized in the material module with a default texture until the user selects a new one using the user interface.

Uniforms objects and samplers must have different binding decoration numbers specified in GLSL using the layout qualifier.


\subsection{Material Module}
A Material is a combination of a vertex and a fragment shader, plus the values of the uniform variables defined in those shaders.

The material module is responsible for parsing the shaders' SPIR-V codes and check uniform names. The code analysis is done using a third-party library, SPIRV-Cross, a tool and library for performing reflection on SPIR-V and disassembling SPIR-V back to high level languages. After the analysis, a description of the requested uniform data is created in order to allow the creation of the uniform buffer object later on. It also parses the image samplers and the attributes requested by the shader code.

The material module also deals with many aspects of the Vulkan API, like creating a descriptor pool, a descriptor set layout, a pipeline layout and a pipeline object.


\subsection{Geometry Module}
Every geometry object has standard transformation parameters (translation,
rotation and scale), and an assigned material.

When a material is assigned, the parsed vertex attributes in the material are
used to create a vertex buffer object, an index buffer object (only if the
geometry has vertex indices data). The parsed uniform data description in the
material is also used by the geometry module to create a uniform buffer object.

This module also deals with Vulkan directly, allocating a descriptor set from
the assigned material's descriptor pool and recording drawing commands in
command buffers.


\subsection{Texture Module}
The texture module deals with Vulkan in order to pass image data for the
shaders. This means creating Vulkan objects for an image, image view and
sampler.

Images are loaded by Godot Engine, the engine used to create the user interface.


\subsection{Camera Module}
A simple module to model a virtual camera. It has a position, target and up vector for reference and an angle for the field of view of the projection.

