% como parametros devem ser passados o titulo e as palavras-chave na outra língua, separadas por vírgulas

\begin{englishabstract}
{ Uma Ferramenta Interativa para o Aprendizado e Exploração de Programação de Shaders }
{ Electronic document preparation. \LaTeX. ABNT. UFRGS }

    O aprendizado de programação de shaders comumente requer que o novato crie uma aplicação hospedeira completa, carregando manualmente modelos poligonais e texturas e lidando com detalhes da API gráfica. Esse processo pode ser bastante desencorojador, desviando a atenção da programação de shaders para o desenvolvimento da infraestrutura da aplicação. Para aliviar esse trabalho, ambientes de desenvolvimento de shaders foram criados por fabricantes de GPUs, como o FX Composer da NVidia e o RenderMonkey da AMD, mas ambos foram descontinuados há bastante tempo, e não oferecem suporte a linguagens modernas de shading. Recursos disponíveis on-line, como Shadertoy e Shdr, podem ser ferramentas valiosas para ajudar no aprendizado de programação de shaders, mas costumam ser difíceis de utilizar, possuem um conjunto de funcionalidades limitado e/ou não oferecem documentação adequada para iniciantes. Nós apresentamos um ambiente completo para o aprendizado e exploração de programação de shaders utilizando GLSL. Nosso ambiente, desevnolvido usando Vulkan, é capaz de suprir as necessidades de usuários iniciantes bem como experientes. Sua interface amigável permite que o usuário carregue modelos 3D e imagens sem dificuldade, além de modificar variáveis e texturas, aplicando as mudanças em tempo real.

\end{englishabstract}
