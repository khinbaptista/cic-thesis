% As a general rule, do not put math, special symbols or citations
% in the abstract
\begin{abstract}
Learning shader programming often requires the novice to create an entire host application, having to manually load polygonal models and textures, and to deal with graphics API details. This process can be quite discouraging, deviating one's attention from actual shader programming  to the development of application infrastructure. To alleviate this load, shader development environments have been created by GPU manufacturers, such as NVidia's FX Composer  and AMD's RenderMonkey, but both have long been discontinued, not supporting modern shading languages. Available  on-line resources, like Shadertoy and Shdr,  can be valuable tools in assisting the learning of shader programming, but are often difficult to use, have limited feature sets, and/or lack proper documentation to get beginners  started. We present a complete environment for learning and exploring shader programming using GLSL. Our environment, developed using Vulkan, can suit the needs of  both beginners and advanced users. Its user-friendly interface allows one to effortlessly load 3D models and images and modify variables and textures, applying the changes in real time. 	
	
%In order to learn shader programming, students often have to create the entire
%host application to manually load 3-dimensional meshes and images, and deal
%with specific graphic API details. This process can easily distract the
%student, deviating attention from the actual shader programming. There are
%tools to assist these tasks, such as NVidia's FX Composer - which has been
%discontinued since 2008 - and some online resources, like Shadertoy and Shdr,
%but those are often difficult to use, have a limited feature set, and/or lack
%documentation to get beginners started.
%
%In this work we built an easy-to-use desktop tool which provides the full
%environment required for shader programming from beginner to advanced, with a
%user-friendly interface which allows students to effortlessly load 3D models
%and images and modify variables and textures while the program is running,
%applying the changes in real time.

\end{abstract}

% no keywords
