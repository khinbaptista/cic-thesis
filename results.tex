\chapter{Results}
The application allows the user to create and visualize different shaders in multiple objects, define an arbitrary number of parameters for each material and also use multiple textures to create the desired visuals.

In figure \ref{fig:phong_sphere}, we see a sphere illuminated using the Phong illumination model and shading (illumination is calculated per pixel). 

\begin{figure}
    \centering
    \caption{Phong shading}
    \includegraphics[width = 7cm]{"results/phong"}
    \label{fig:phong_sphere}
\end{figure}

\begin{figure}
    \centering
    \caption{Phong shading with normal mapping}
    \includegraphics[width = 7cm]{"results/normal"}
    \label{fig:phong_normal_sphere}
\end{figure}

\begin{figure}
    \centering
    \caption{Multiple objects with different materials}
    \includegraphics[width = 13cm]{"results/meshes-multi-material"}
    \label{fig:scene_1}
\end{figure}

\begin{figure}
    \centering
    \caption{An example scene}
    \includegraphics[width = 15cm]{"results/scene"}
    \label{fig:scene_2}
\end{figure}

\begin{figure}
    \centering
    \caption{Dragon closeup}
    \includegraphics[width = 15cm]{"results/dragon-closeup"}
    \label{fig:dragon_closeup}
\end{figure}

\begin{figure}
    \centering
    \caption{Torus closeup}
    \includegraphics[width = 15cm]{"results/torus-closeup"}
    \label{fig:torus_closeup}
\end{figure}

\begin{figure}
    \centering
    \caption{Sphere closeup}
    \includegraphics[width = 15cm]{"results/sphere-closeup"}
    \label{fig:sphere_closeup}
\end{figure}
