\chapter{Conclusion}
We presented a complete environment for learning and exploring shader
programming using GLSL. Our environment, developed using Vulkan, can suit the needs of both beginners and advanced users. Its user-friendly interface allows one to effortlessly load 3D models and images and modify variables and textures, applying the changes in real time. This was also an important learning experience. Developing a real application using  Vulkan provided the opportunity to learn such new graphics API. 

In order to provide a context for this work, the thesis also reviewed the concept of graphics pipeline and its stages, especially the programmable stages, all of which are key concepts in the field of computer graphics. We also introduced the tool used to build our application, Godot Engine, and the technology called GDNative, used in our application to integrate the Vulkan host application and the control window. We also talked about graphics APIs and how APIs have different characteristics that the developer must be aware of.

Then, we presented the details of the Vulkan API, used throughout this work. As we could see, this API burdens the developers with every detail of the graphics pipeline and configuration. While this empowers programmers to tailor the application that is most optimized, it can also be very difficult to do so.

We described the architecture and how our tool works from the inside, explaining each of its modules and how they work to build the best user experience possible. The user interface is also discussed in depth, showing the features from a user perspective and describing how to use each of them.

The application we created is a tool that allows students to write shader code without having to worry about the underlying graphics API or creating the application infrastructure just to see the shaders in action. The results show we were successful in creating such tool, rendering different scenes which demonstrate the flexibility achieved.

Creating this application has been a great learning exercise of the Vulkan API, and the efforts have resulted in a very interesting piece of software with a real-world use-case scenario, with lots of room to grow. Given the proper treatment, this application, which is currently presented as a proof of concept, could be really useful in computer graphics lectures and workshops, from fundamentals of computer graphics to advanced shader programming.
