\section{User Interface}
In this section we will describe how our application works from a user perspective, explaining the different panels, their responsibilities and how to use them.

\begin{figure}[ht]
    \caption{Shader Tutor control window}
    \begin{center}
        \includegraphics[width = 15cm]{"editor/window"}
    \end{center}
    \label{fig:window}
    \legend{Source: the author}
\end{figure}

\begin{figure}[ht]
    \caption{Panels of the control window}
    \begin{center}
        \includegraphics[width = 15cm]{"editor/ShaderTutorInterfaceBreakdown"}
    \end{center}
    \label{fig:interface_breakdown}
    \legend{Source: the author}
\end{figure}

Figure \ref{fig:interface_breakdown} shows the different panels of Shader Tutor:
\begin{enumerate}
    \item \textbf{Material list}: Lists all the materials loaded. Allows the user to load shaders and compile them, which will prepare the material to be used in the scene;
    \item \textbf{Shader editor}: Source code editor featuring GLSL syntax highlighting, line numbers and other useful features. Also contains a multiline shader compilation status bar to display any errors in the shader code found during compilation;
    \item \textbf{Uniform variables editor}: Uniform variables defined in shader code will be displayed in this panel as widgets which allow the user to change the values of the variables. The changes are applied instantly in the visualization window;
    \item \textbf{Object list}: List of the objects in the scene. Allows the user to create new objects, either from primitives (capsule, cube, cylinder, plane, prism or sphere) or loaded from a file;
    \item \textbf{Inspector}: Displays the transformation applied to the selected object in the object list (translation, rotation and scale) and the material used to display it.
\end{enumerate}

Figure \ref{fig:visualization_window} shows that the visualization window, which displays the scene, is a different window from the control window described above.

\begin{figure}[ht]
    \caption{Visualization and control windows}
    \begin{center}
        \includegraphics[width = 15cm]{"editor/visualization_window"}
    \end{center}
    \label{fig:visualization_window}
    \legend{Source: the author}
\end{figure}

%%%%%%%%%%%%%%%%%%%%%%%%%%%%%%%%%%%%%%%%%%%%%%%%%%
\subsection{Materials and Shaders}
When the application is first executed, a default material is loaded. This is a simple unshaded material that will be assigned to new objects when they are created. This material does not require the geometry to have any any attributes other than position, which ensures this material can be applied to any loaded mesh.

The \textit{create button} (labeled with a plus sign) allows the user to load shaders and create a new material via the \textit{create material} popup window (Figure \ref{fig:create-material}). The material name must be unique and the vertex and fragment shaders source files must have a ".vert" and ".frag" extension, respectively. When the material is created, the contents of the shader files are loaded in the shader editor (Figure \ref{fig:shader-editor}). At this point, the material is not ready for usage yet.

\begin{figure}[h]
    \caption{Create material popup window}
    \begin{center}
        \includegraphics[width = 10cm]{"editor/create-material"}
    \end{center}
    \label{fig:create-material}
    \legend{Source: the author}
\end{figure}

\begin{figure}
    \caption{Shader editor}
    \begin{center}
        \includegraphics[width = 10cm]{"editor/shader-editor"}
    \end{center}
    \label{fig:shader-editor}
    \legend{Source: the author}
\end{figure}

The GLSL source code for the vertex and fragment shaders can be altered in the shader editor, which features syntax highlighting, line numbers, highlight selected word occurrences and a shader compilation status bar.

To compile the shaders and setup the material, the user can click on the execute button in the material list panel (the button with the "play" icon). The shaders are then compiled and loaded in the material. If the compilation is unsuccessful, the status bar, right under the fragment shader editor, displays the error messages. When the compilation succeeds, the uniform variables defined in the shaders' source code are exposed in the material parameters panel (Figure \ref{fig:uniform-editor}).

\begin{figure}
    \caption{Uniform editor}
    \begin{center}
        \includegraphics[width = 10cm]{"editor/uniform-editor"}
    \end{center}
    \label{fig:uniform-editor}
    \legend{Source: the author}
\end{figure}

The editor creates appropriate widgets for each uniform variable defined in the shaders based on their types. Supported types are described in Table \ref{tab:uniform_types}.

\begin{table}
    \centering
    \caption{Recognized types for uniform variables}
    \begin{tabular}{|c|p{10cm}|}
        \hline
        Type name & Widget created \\
        \hline \hline
        \texttt{int} & Single numerical input widget. Accepts values from -10000 to 10000. Fractions are rounded. \\ \hline
        \texttt{uint} & Single numeric input widget. Accepts values between 0 and 10000. Fractions are rounded. \\ \hline
        \texttt{float} & Single numeric input field. Accepts values between -10000.0 and 10000.0. Uses 2 decimal digits. \\ \hline
        \texttt{ivec2} & Two numerical input widgets. Each widget follows the same input rules as \texttt{int}. \\ \hline
        \texttt{uvec2} & Two numerical input widgets. Each widget follows the same input rules as \texttt{uint}. \\ \hline
        \texttt{vec2} & Two numerical input widgets. Each widget follows the same input rules as \texttt{float}. \\ \hline
        \texttt{ivec3} & Three numerical input widgets. Each widget follows the same input rules as \texttt{int}. \\ \hline
        \texttt{uvec3} & Three numerical input widgets. Each widget follows the same input rules as \texttt{uint}. \\ \hline
        \texttt{vec3} & Three numerical input widgets. Each widget follows the same input rules as \texttt{float}. \\ \hline
        \texttt{vec4} & Color picker widget \\
        \hline
    \end{tabular}
    \label{tab:uniform_types}
\end{table}

%%%%%%%%%%%%%%%%%%%%%%%%%%%%%%%%%%%%%%%%%%%%%%%%%%
\subsection{Objects}

The plus button on the objects panel allows the user to create a new object, selecting a primitive geometry from a list or loading a custom mesh file (in Wavefront ".obj" format). If the file defines multiple groups inside the object, the groups will be merged and loaded as a single group. The geometry primitives provided are: capsule, cube, cylinder, plane, prism and sphere.

Newly created objects are given a name based on their shape (if the object is a primitive) or their file name (if it was created from a mesh file), and their creation order. New objects are assigned the default material upon creation, and instantly appear in the scene (Figure \ref{fig:default_material}).

\begin{figure}
    \caption{Dragon mesh with the default material}
    \begin{center}
        \includegraphics[width = 10cm]{"editor/default_material"}
    \end{center}
    \label{fig:default_material}
    \legend{Source: the author}
\end{figure}

When an object is selected, their transformation and currently assigned material are displayed on the inspector panel, below the material parameters panel (Figures \ref{fig:uniform-editor} and \ref{fig:inspector}). Changes made to the inspector and the material parameters take effect immediately.

\begin{figure}
    \caption{Inspector}
    \begin{center}
        \includegraphics[width = 10cm]{"editor/inspector"}
    \end{center}
    \label{fig:inspector}
    \legend{Source: the author}
\end{figure}

\begin{figure}
    \caption{Available materials appear in the inspector drop-down menu}
    \begin{center}
        \includegraphics[width = 6cm]{"editor/material_selection"}
    \end{center}
    \label{fig:material_selection}
    \legend{Source: the author}
\end{figure}

Table \ref{tab:navigation_control} shows the available commands to navigate through the scene.

\begin{table}[h]
    \centering
    \caption{Scene navigation controls}
    \begin{tabular}{|c|p{8cm}|}
        \hline
        Input & Action \\
        \hline \hline
        Mouse right click & When pressed, enables movement \\ \hline
        Mouse movement & When movement is enabled, rotates the camera around its own center, looking around \\ \hline
        'W' key & Move the camera forward, relative to its current orientation \\ \hline
        'A' key & Move the camera to the left, relative to its current orientation \\ \hline
        'S' key & Move the camera backward, relative to its current orientation \\ \hline
        'D' key & Move the camera to the right, relative to its current orientation \\
        \hline
    \end{tabular}
    \label{tab:navigation_control}
\end{table}
