\subsection{Geometry Module}
The geometry module represents an object in the scene. It holds all the vertices attributes (vertex position, normal, tangent, texture coordinates and indices, if available); a transformation, which defines the rotation, scale and translation applied to the object; and a reference to a material object which will be used to render the geometry in the scene.

When an object is added to the scene, it must be filled with vertex data and then a material has to be assigned to the object. Then, the object will get a list of the required vertex attributes from the material, and create the vertex buffer accordingly. If the vertex attributes include index data, the index buffer is created, too. Then, the uniform buffers are created, based on the uniform buffer objects in the material assigned.

For the uniform buffer to be passed to the shaders, we need to allocate a descriptor set from the pool defined in the material assigned to the geometry.

\subsubsection{Recording Drawing Commands}
Since all drawing operations are recorded in advance in Vulkan, this module also provides a method for recording the drawing commands in a command buffer previously initialized.

\begin{itemize}
    \item \texttt{VkCmdBindPipeline}: Binds the graphics pipeline of the material applied to the object;
    \item \texttt{VkCmdBindVertexBuffer}: Binds the vertex buffer of the object, which contains the vertex attributes;
    \item \texttt{VkCmdBindDescriptorSets}: Binds the descriptor set which contains the uniform variables data;
    \item \texttt{VkCmdBindIndexBuffer}: If the object presents index data, binds the buffer containing the indices;
    \item \texttt{VkCmdDrawIndexed}: Draw the object using the index buffer;
    \item \texttt{VkCmdDraw}: Draw object without indices.
\end{itemize}

\subsubsection{Updating Uniform Buffers}
The function to update uniform buffers is called by the Vulkan application module, and expects a structures containing certain uniform values that are not dependent on the object, such as the camera's view and projection matrices. The other matrices are calculated and loaded into the appropriate buffers.

The material can define arbitrary uniform variables, and those must be updated as well. The material holds an array of dictionaries that describe each of the user-defined variables, their types, sizes, bindings, offsets and values. With this information, our application is able to load the variable value into the correct buffer in the correct offset.
