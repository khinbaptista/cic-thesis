\subsection{Texture Module}
The texture module handles the passage of image data for the shaders. This means creating an image, allocating device memory for the image, creating the image view and the sampler.

Image files are loaded by Godot Engine, the tool used to create the user interface, and their data is passed to the texture module. Our application then has to pass this data into the device memory bound to the image. However, there is no way to write directly to the image memory, which forces us to use a \textit{staging buffer}. The staging buffer can be mapped to CPU memory and accessed by our application. In order to transfer this data to the image, the image must have its layout transitioned to optimal for transfer destination. This is done using a single time command buffer, which is allocated, recorded, executed and freed, sequentially. Single time command buffers are managed by the Vulkan application module. Once the image is ready to receive a data transfer and the staging buffer is loaded, we can use another single time command buffer to copy the buffer contents into the image. After this transfer is complete, the image must be transitioned again to a layout optimal for shader read only. Then, the staging buffer can be destroyed, and the device memory can be freed.
