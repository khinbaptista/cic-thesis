\section{User Interface}
First, the user will use their favorite editor of choice to write the GLSL shaders, saving the vertex and fragment shaders in separate files, ending in ".vert" and ".frag", respectively. Then, in our application, the user will load this file as a new shader, and give it a name (the shader name defaults to the file name, and must be unique). If the shader compilation fails, the shader is not created, and the compilation error message is displayed in the console. Having created both vertex and fragment shaders successfully, the user can then create a new material, and give it a name, which has to be unique. After the material is created, clicking on it will reveal a panel where the user can change the shaders used in the material. On newly created materials, the default shaders will be used. After choosing the shaders from the drop-down menus, the "apply" button will update the material with the new shaders, and the uniforms defined in the shaders will appear in the editor on the rightmost panel.

Objects can be created in the <IForgotTheName> tab. Currently, only some geometric primitives can be created (rather than loading custom objects), like cubes, spheres, capsules and planes. Objects are given a name based on their shape (the geometry itself) and their creation order. When an object is selected, their transform is exposed in the rightmost panel, so the user can change the object's position, rotation and scale, and also change the material. Upon creation, the default material will be used.
